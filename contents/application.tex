\section{量子算法应用:以 ACM 班程序设计作业为例}

\subsection{背景与题目}

在量子算法发明之初,以 Deutsch-Jozsa 为例的多种算法应用范围都相对局限,对于多数实际问题都无能为力或无法带来显著变化。但随着量子算法的逐步发展,许多经典计算机能解决的问题都能被量子计算机找到更优的方案。下面以上海交通大学 ACM 班 2022 级程序设计大作业的一道题\ \href{https://acm.sjtu.edu.cn/OnlineJudge/problem?problem_id=1754}{P1754 int2048-乘法速度测试} 为例,讲解量子算法的简单应用。

题目的大意为:给定若干个大整数,求它们的乘积,所有运算结果范围在 $[1,10^{200000}]$ 内。为简化题意,题目可以改写为,给定正整数 $a,b$ 满足 $a,b\le 10^{100000}$,求 $a\times b$。一种朴素的做法是直接模拟竖式乘法,其时间复杂度为 $n^2$(其中 $N=\lceil\log_k\max{a,b}\rceil$,$k$ 为运算过程中的进制)。进一步地,可以考虑整数的多项式表示:对于一个 $k$ 进制整数 \begin{align}
    a=\overline{a_1\cdots a_N}=\sum_{i\in[N]}a_i k^{N-i}
\end{align}
考虑多项式 \begin{align}
    A(z)=\sum_{i\in[N]}a_i z^{N-i}
\end{align}
同理可构造出 $b$ 对应的多项式 $B(z)$,那么令 \begin{align}
    C(z)=A(z)B(z)=\sum_{i\in[2N]}\left(\sum_{j\in[i]}a_jb_{i-j}\right)z^i
\end{align}
在经典计算机中,利用快速傅里叶变换(Fast Fourier Transform)与其逆变换可以在 $O(N\log N)$ 的时间内求出对应的多项式 $C$\cite{cooley1965algorithm},进行进位就得到了 $a\times b$。接下来,尝试使用量子计算机得到一种更加高效的做法。

\subsection{前置知识}

在离散傅里叶变换(Discrete Fourier Transform,DFT)中,我们熟知对多项式 $f(x)$ 进行点值 $(\omega_N^k,y_k)$ 求值的方法 \begin{align}
    y_k=\frac{1}{\sqrt{N}}\sum_{j=0}^{N-1}x_j\omega_N^{jk}.
\end{align}
在量子计算机中,我们同样希望进行相同的变换,使得 $N-1$ 维量子态 $\ket{X}=\sum_{j=0}^{N-1}x_j\ket{j}$,经过变换得到 $\ket{Y}=\sum_{k=0}^{N-1}y_k\ket{k}$,其中 $y_k$ 与 $x_k$ 的关系满足上式,$\ket{j},\ket{k}$ 表示其二进制分解所得结果的张量积。代入可得 \begin{align}
    \ket{Y}=\sum_{k=0}^{N-1}y_k\ket{k}=\sum_{k=0}^{N-1}\left(\frac{1}{\sqrt{N}}\sum_{j=0}^{N-1}x_j\omega_N^{jk}\ket{k}\right)=\sum_{j=0}^{N-1}x_j\left(\frac{1}{\sqrt{N}}\sum_{k=0}^{N-1}\omega_N^{jk}\ket{k}\right)
\end{align}
对比 $\ket{X},\ket{Y}$ 的形式,发现变换的实质就是一次基变换 \begin{align}
    \ket{j}\to\frac{1}{\sqrt{N}}\sum_{k=0}^{N-1}\omega_N^{jk}\ket{k}\label{eq:qft-change-basis}
\end{align}
两组标准正交基的变换可改写为酉变换 \begin{align}
    U_{\text{QFT}} = \frac{1}{\sqrt{N}}
    \begin{bmatrix}
        1      & 1               & 1                         & \cdots & 1                      \\
        1      & \omega_N        & \omega_N ^{2}             & \cdots & \omega_N ^{N-1}        \\
        1      & \omega_N ^{2}   & \omega_N ^{4}             & \cdots & \omega_N ^{2(N-1)}     \\
        1      & \omega_N ^{3}   & \omega_N ^{6}             & \cdots & \omega_N ^{3(N-1)}     \\
        \vdots & \vdots          & \vdots                    & \ddots & \vdots                 \\
        1      & \omega_N ^{N-1} & \omega_N ^{(N-1)\times 2} & \cdots & \omega_N ^{(N-1)(N-1)}
    \end{bmatrix}
\end{align}
这样就完成了一个满足量子计算要求的傅里叶变换算法,也即量子傅里叶变换(Quantum Fourier Transform,QFT)。

\subsection{电路设计}

上面的内容已经完成了量子傅里叶变换的算法设计,但得到了这样的酉阵,还不能直接设计出所需的量子电路,因此,在这里将式 \ref{eq:qft-change-basis} 进行进一步演化:考虑直接对 $\omega_{2^n}^{jk}=\exp(\frac{2\pi\mathrm{i} jk}{2^n})$ 指数中的 $\frac{k}{2^n}$ 进行二进制小数分解,也即对 $k$ 进行二进制分解,得到 \begin{align}\begin{aligned}
        \ket{j}\to\frac{1}{\sqrt{N}}\sum_{k=0}^{N-1}\omega_N^{jk}\ket{k}
         & =2^{-\frac{n}{2}}\sum_{l\in[n],k_l\in\{0,1\}}\exp\left(2\pi\mathrm{i} j\sum_{l\in[n]}k_l2^{-l}\right)\ket{k_1,\cdots,k_n}                  \\
         & =2^{-\frac{n}{2}}\bigotimes_{l\in[n]}\left(\sum_{k_l\in[0,1]}\exp(2\pi\mathrm{i} jk_l2^{-l})\ket{k_l}\right)                               \\
         & =2^{-\frac{n}{2}}\bigotimes_{l\in[n]}\left(\ket{0}+\exp(2\pi\mathrm{i} j2^{-l})\ket{1}\right)                                              \\
         & =2^{-\frac{n}{2}}\bigotimes_{l\in[n]}\left(\ket{0}+\exp\left(2\pi\mathrm{i}\left\{\frac{j}{2^l}\right\}\right)\ket{1}\right)\label{eq:qft}
    \end{aligned}\end{align}
这里的 $\{\frac{j}{2^l}\}$ 表示取 $\frac{j}{2^l}$ 的小数部分,由于 $e^{2\pi\mathrm{i}}=1$,因此等式是成立的。为了方便构建电路,记 \begin{align}
    \frac{j}{2^n}=\overline{0.j_1 j_2\cdots j_n}
\end{align}
那么式 \ref{eq:qft} 就可以写作 \begin{align*}
    2^{-\frac{n}{2}}\bigotimes_{l\in[n]}\left(\ket{0}+\exp(2\pi\mathrm{i}\overline{0.j_{n-l+1}\cdots j_{n}})\ket{1}\right)
\end{align*}

这种张量积意义下的量子电路就可以通过旋转门 $R_k$ 加以控制生成,\begin{align}
    R_k=\begin{bmatrix}
            1 & 0                                           \\
            0 & \exp\left(\frac{2\pi\mathrm{i}}{2^k}\right)
        \end{bmatrix}
\end{align}
特殊地,$H$ 可以直接看作自身受控的 $R_1$ 门,这是因为 $H\ket{0}=\ket{+},H\ket{1}=\ket{-}$

例如,考虑输入态 $\ket{j_1,\cdots,j_n}$,将其首位通过 $H$ 门将得到 \begin{align}
    \frac{1}{\sqrt{2}}\left(\ket{0}+\exp(2\pi\mathrm{i}\overline{0.j_1})\ket{1}\right)\ket{j_2,\cdots,j_n}
\end{align}
随后,每次让首位通过受 $j_k$ 控的 $R_k$ 门,相位都将增加 $\frac{j_k\pi}{2^{k-1}}$,最终显然会得到 \begin{align}
    \frac{1}{\sqrt{2}}\left(\ket{0}+\exp(2\pi\mathrm{i}\overline{0.j_1\cdots j_n})\ket{1}\right)\ket{j_2,\cdots,j_n}
\end{align}
依此类推,每一位需要的状态都可以这样进行制备,这样就得到了量子电路\footnote{参考了 Michael Charemza 的 \LaTeX 代码 \cite{charemza2006examples}。}

\begin{figure}[H]
    \centering
    \begin{minipage}{12cm}
        \centering
        \Qcircuit @C=1em @R=1em {
        % \begin{bmatrix}
        \lstick{\ket{j_{n}}}   & \gate{\hbox{$H$}} & \gate{\hbox{$R_2$}} & \gate{\hbox{$R_3$}} & \qw    & \gate{\hbox{$R_n$}} & \qw               & \qw                 & \qw    & \qw                     & \qw               & \qw                 & \qw    & \qw                     & \qw               & \rstick{\ket{y_1}} \qw   \\
        \lstick{\ket{j_{n-1}}} & \qw               & \ctrl{-1}           & \qw                 & \qw    & \qw                 & \gate{\hbox{$H$}} & \gate{\hbox{$R_2$}} & \qw    & \gate{\hbox{$R_{n-1}$}} & \qw               & \qw                 & \qw    & \qw                     & \qw               & \rstick{\ket{y_2}} \qw   \\
        \lstick{\ket{j_{n-2}}} & \qw               & \qw                 & \ctrl{-2}           & \qw    & \qw                 & \qw               & \ctrl{-1}           & \qw    & \qw                     & \gate{\hbox{$H$}} & \gate{\hbox{$R_2$}} & \qw    & \gate{\hbox{$R_{n-2}$}} & \qw               & \rstick{\ket{y_3}} \qw   \\
        \lstick{\vdots }       &                   &                     &                     & \ddots &                     &                   &                     & \ddots &                         &                   &                     & \ddots &                         &                   & \rstick{\vdots }         \\
        \lstick{\ket{j_{1}}}   & \qw               & \qw                 & \qw                 & \qw    & \ctrl{-4}           & \qw               & \qw                 & \qw    & \ctrl{-3}               & \qw               & \ctrl{-2}           & \qw    & \ctrl{-2}               & \gate{\hbox{$H$}} & \rstick{\ket{y_{n}}} \qw
        % \end{bmatrix}
        }
    \end{minipage}
    \label{fig:qft-circuit}
\end{figure}

于是,这样就可以完成量子傅里叶变换,将两个乘数的傅里叶变换点值相乘,就得到了积的多项式点值,进而可以利用量子运算可逆的性质方便地进行量子傅里叶逆变换,只需要将刚刚的操作完全反转即可,其电路为

\begin{figure}[H]
    \centering
    \begin{minipage}{12cm}
        \centering
        \Qcircuit @C=0.6em @R=1em {
        % \begin{bmatrix}
        \lstick{\ket{j_{n}}}   & \qw                           & \qw                                       & \qw     & \qw                                   & \qw                           & \qw                                       & \qw     & \qw                                   & \qw                           & \gate{\hbox{\scriptsize$(R_n)^{-1}$}} & \gate{\hbox{\scriptsize$(R_{n-1})^{-1}$}} & \qw     & \gate{\hbox{\scriptsize$(R_2)^{-1}$}} & \gate{\hbox{\scriptsize $H$}} & \rstick{\ket{y_1}} \qw   \\
        \lstick{\ket{j_{n-1}}} & \qw                           & \qw                                       & \qw     & \qw                                   & \qw                           & \gate{\hbox{\scriptsize$(R_{n-1})^{-1}$}} & \qw     & \gate{\hbox{\scriptsize$(R_2)^{-1}$}} & \gate{\hbox{\scriptsize $H$}} & \qw                                   & \qw                                       & \qw     & \qw                                   & \qw                           & \rstick{\ket{y_2}} \qw   \\
        \lstick{\ket{j_{n-2}}} & \qw                           & \gate{\hbox{\scriptsize$(R_{n-2})^{-1}$}} & \qw     & \gate{\hbox{\scriptsize$(R_2)^{-1}$}} & \gate{\hbox{\scriptsize $H$}} & \qw                                       & \qw     & \qw                                   & \qw                           & \qw                                   & \qw                                       & \qw     & \qw                                   & \qw                           & \rstick{\ket{y_3}} \qw   \\
        \lstick{\vdots }       &                               &                                           & \iddots &                                       &                               &                                           & \iddots &                                       &                               &                                       &                                           & \iddots &                                       &                               & \rstick{\vdots }         \\
        \lstick{\ket{j_{1}}}   & \gate{\hbox{\scriptsize $H$}} & \ctrl{-2}                                 & \qw     & \ctrl{-2}                             & \qw                           & \ctrl{-3}                                 & \qw     & \ctrl{-3}                             & \qw                           & \ctrl{-4}                             & \ctrl{-4}                                 & \qw     & \qw                                   & \qw                           & \rstick{\ket{y_{n}}} \qw
        % \end{bmatrix}
        }
    \end{minipage}
    \label{fig:iqft-circuit}
\end{figure}

在此不再列出其电路。在每一个过程中,用到了 $\frac{n(n+1)}{2}$ 个量子门,时间复杂度为 $O(n^2)=O(\log^2 N)$,算法效率较经典计算机中的快速傅里叶变换有了显著的提升。利用 IBM 的 Python 宏包 Qiskit 可以在 Jupyter Notebook 中完成这样的量子乘法器,笔者的代码\footnote{参考了 qiskit-tutorials 中的实现代码\cite{Qiskit} 与《20 行 Python 代码说清量子霸权\cite{马超20行Python代码说清量子霸权}。}在 \href{https://github.com/Conless/quantum-computing-intro/blob/main/src/qft.ipynb}{GitHub} 开源。
