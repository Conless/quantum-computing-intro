\section{引入与简介}

量子计算,指一种运用量子力学现象,例如量子叠加、量子干涉与量子纠缠进行计算的方式,是过去的数十年中兴起的一种全新计算方式,在近几年中也是一个在各类科普中高频出现的热点词汇。下面的内容从经典计算机讲起,简要地介绍量子计算的发展历史与其重要意义\footnote{部分内容参考了维基百科词条与 \textit{Quantum Computation and Quantum Information}\cite{nielsen2002quantum}}。

引入量子计算的初衷,就是为了打破经典计算机出现的性能瓶颈。经典计算机的发展领先于量子计算机约半个世纪;Alan Turing 在 1936 年发表的论文中提出了图灵机(Turing Machine)的模型\cite{turing1936computable},宣告了现代计算机科学的诞生。Turing 证明了存在一台通用图灵机,即任何可以在个人电脑上执行的算法,都可以在这台图灵机上完成,这个论断被称为 Church-Turing 命题。随后,von Neumann 设计出了这样的理论模型,用实际元件实现了通用图灵机的全部功能,在随后的几十年里,个人计算机的发展也一直沿用 von Neumann 架构,其发展速度遵从 1965 年 Gordon Moore 所概括的 Moore 定律,即集成电路中单位面积的晶体管数量,以及与之相对应的,计算机计算速度,大约每两年增长一倍\cite{moore1965cramming}。

自 Moore 定律提出以来,经典计算机硬件发展速度都近似地遵从于该定律;但进入 21 世纪以来,Moore 定律的有效性逐渐下降,许多研究人员认为其将在 21 世纪的前 20 年终结,著名芯片企业,Nvidia 公司的首席执行官 Jensen Huang 就于日前宣称,Moore 定律已死\cite{moorelawdead}。其中的重要原因在于传统半导体原件在栅极线宽较小时,可能会出现量子隧穿等效应\cite{kumar2015fundamental}。

解决 Moore 定律最终失效的一个可能方案是采用不同的计算模式,量子计算就是其中一种。量子计算这一概念于 1980 年由物理学家 Paul Benioff 首次提出\cite{benioff1980computer}。随后,Feymann 指出,在经典计算机上有效地模拟量子系统的演化似乎是不可能的,量子计算机可能可以模拟经典计算机无法做到的事情\cite{feynman1981simulating},并引入了早期版本的量子电路符号\cite{feynman1986quantum}。1985 年,David Desutsch 提出,能否用量子力学原理推导出更强的 Church-Turing 命题,并引导出了现代量子计算机的概念\cite{deutsch1985quantum}。他举了一个简单的例子(见节 \ref{sec:deutsch}),表明量子计算机的计算能力确实超过了经典计算机。

在随后的十年里,量子算法相关研究不断涌现出新的成果。1994 年,Peter Shor 提出了一种新的量子算法,可以有效地解决大整数的质因数分解问题\cite{shor1994algorithms},这在经典计算机上被认为是不可解(难以在多项式时间内解决)的;Shor 算法的出现一度让基于质因数分解的 RSA 加密算法的安全性受到威胁\cite{mermin2006breaking}。1996 年,Lov Grover证明了,在非结构化搜索空间进行搜索的问题也可以通过量子计算机加速\cite{grover1996fast},这种搜索方法的广泛适用性引起了人们对 Grover 算法的相当关注。

由此可以看出,量子计算与量子算法可以对许多运用经典计算机难以解决的问题进行加速。如今,量子计算领域依旧处于蓬勃发展之中。在下面的章节中,量子计算的相关基础理论将会被介绍。