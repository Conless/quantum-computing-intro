\section{思考与展望}

在过去的一段时间里,笔者对量子计算这一领域进行了初步的了解与学习。从一开始,对量子的叠加性特点与对量子比特进行数学计算方法的困惑,到后来,通过检索资料、求助同学等方式,逐渐理解了量子计算的原理,并通过与经典计算机的对比理解了量子计算的诸多特点;下面简要概括我的关于此的一些思考。

在深入学习之前,我一直不理解的一个问题是:如果说量子计算可以直接由矩阵与向量等传统数学操作完成,那么它和经典计算机中直接进行数学运算的区别在哪里?为什么可以带来效率的提升?可以考虑一个 $n$ 量子比特门,对于一个酉变换,它能在 $O(n)$ 的时间内完成量子比特的变换,表面上看仅仅是改变了 $n$ 个比特位上的概率分布,实际上却同时改变了关于一组 $2^n$ 个标准正交基的线性组合。这就意味着,如果初态与变换选取得当,量子计算就可以用 $O(n)$ 的时间完成 $O(2^n)$ 的工作,这样的案例已经在节 \ref{sec:deutsch} 中用 Deustch-Jozsa 算法展示了。

在了解了简单的几种量子算法后,我产生了进一步的好奇:目前产生了哪些类型的量子算法,它们能解决哪些经典计算机科学中的问题?\textit{ Quantum Computation and Quantum Information}\cite{nielsen2002quantum} 中对现行的量子算法进行了这样的分类:基于傅里叶变换的量子算法,例如 Shor 的素因子分解和离散对数算法;量子搜索算法,例如大名鼎鼎的 Grover 搜索;以及量子模拟算法。此前一段时间里我主要学习的是第一类量子算法。在尝试学习后面两种算法时,遇到的瓶颈在于对复向量空间的变换还没有建立起一个很好的几何直观,因此对例如 Grover 搜索中的 Amplitude Amplification 与迭代操作并没有理解得很透彻,希望在本学期结束后继续这一领域的学习。

展望量子计算领域的未来发展,业界许多人认为该领域的前景具有很强的不确定性。很大程度上这是因为量子计算机的物理实现进展较为缓慢,理论领域的许多算法在工程上依旧无法进行实现。物理学家David DiVincenzo列出了实用量子计算机的这些要求\cite{divincenzo2000physical}:物理上可扩展以增加量子比特的数量、可以初始化为任意值的量子比特、比退相干时间更快的量子门、量子通用门的嵌套与易于读取的量子比特。为了实现量子计算机,人们也提出了多种方案,目前较为热门的有超导量子计算、量子点量子计算等。回到理论领域,尽管现在已经有许多高效的量子算法,但它们的应用范围还是相对狭窄,科学家们一直在致力于拓展量子算法的应用范围,并降低其不确定性;希望在不远的未来,越来越多的实用量子算法能被发现并投入应用。
