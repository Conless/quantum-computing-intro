\usepackage[left=2.50cm, right=2.50cm, top=2.50cm, bottom=2.50cm]{geometry}

\usepackage{fontspec}

\usepackage{qcircuit}

\usepackage[backref]{hyperref}

\usepackage{amsmath, amsfonts, amssymb, amsthm, extarrows}

\numberwithin{figure}{section}

\usepackage{enumitem}

\usepackage{xcolor}

\usepackage{graphicx}

\usepackage{subfigure}

\usepackage{url}

\usepackage{bm}

\usepackage{multirow}

\usepackage{booktabs}

\usepackage{epstopdf}

\usepackage{epsfig}

\usepackage{listings}

\usepackage{longtable}

\usepackage{supertabular}

\usepackage{algorithm}

\usepackage{algorithmic}

\usepackage{changepage}

\lstset{
	basicstyle=\small\ttfamily,	% 基本样式
		keywordstyle=\color{blue}, % 关键词样式
		commentstyle=\color{gray!50!black!50},   	% 注释样式
		stringstyle=\rmfamily\slshape\color{red}, 	% 字符串样式
	backgroundcolor=\color{gray!5},     % 代码块背景颜色
	frame=leftline,						% 代码框形状
	framerule=12pt,%
		rulecolor=\color{gray!90},      % 代码框颜色
	numbers=left,				% 左侧显示行号往左靠, 还可以为right ,或none,即不加行号
		numberstyle=\footnotesize\itshape,	% 行号的样式
		firstnumber=1,
		stepnumber=1,                  	% 若设置为2,则显示行号为1,3,5
		numbersep=7pt,               	% 行号与代码之间的间距
	aboveskip=.25em, 			% 代码块边框
	showspaces=false,               	% 显示添加特定下划线的空格
	showstringspaces=false,         	% 不显示代码字符串中间的空格标记
	keepspaces=true, 					
	showtabs=false,                 	% 在字符串中显示制表符
	tabsize=2,                     		% 默认缩进2个字符
	captionpos=b,                   	% 将标题位置设置为底部
	flexiblecolumns=true, 			%
	breaklines=true,                	% 设置自动断行
	breakatwhitespace=false,        	% 设置自动中断是否只发生在空格处
	breakautoindent=true,			%
	breakindent=1em, 			%
	title=\lstname,				%
	escapeinside=``,  			% 在``里显示中文
	xleftmargin=1em,  xrightmargin=1em,     % 设定listing左右的空白
	aboveskip=1ex, belowskip=1ex,
	framextopmargin=1pt, framexbottommargin=1pt,
        abovecaptionskip=-2pt,belowcaptionskip=3pt,
	% 设定中文冲突,断行,列模式,数学环境输入,listing数字的样式
	extendedchars=false, columns=flexible, mathescape=true,
	texcl=true,
	fontadjust
}

{
    \theoremstyle{definition}
    \newtheorem{axiom}{\indent 公理}
    \newtheorem{theorem}{\indent 定理}[section]
    \newtheorem{lemma}[theorem]{\indent 引理}
    \newtheorem{proposition}[theorem]{\indent 命题}
    \newtheorem{corollary}[theorem]{\indent 推论}
    \newtheorem{definition}[theorem]{\indent 定义}
    \newtheorem*{solution}{\indent 解}
    \newtheorem{example}{\indent 例}[section]\theoremstyle{definition}
    \newtheorem*{axiom*}{\indent 公理}
    \newtheorem*{theorem*}{\indent 定理}
    \newtheorem*{lemma*}{\indent 引理}
    \newtheorem*{proposition*}{\indent 命题}
    \newtheorem*{corollary*}{\indent 推论}
    \newtheorem*{definition*}{\indent 定义}
    \newtheorem*{example*}{\indent 例}
    \renewcommand{\proofname}{\indent\bf 证明}
}

\renewcommand{\proofname}{\indent\bf 证明}
\newcommand{\bra}[1]{\langle#1|}
\newcommand{\ket}[1]{|#1\rangle}
\newcommand{\inner}[2]{\langle#1|#2\rangle}
\newcommand{\tensor}{\otimes}
\newcommand{\xor}{\oplus}

\newcommand*{\dif}{\mathop{}\!\mathrm{d}}

\setmainfont{Times New Roman}

\usepackage{xpatch}
\makeatletter
\xpatchcmd{\@thm}{\thm@headpunct{.}}{\thm@headpunct{}}{}{}
\makeatother


{
    \theoremstyle{plain}
    \newtheorem*{think}{\indent 思考}
    \newtheorem*{note}{\indent 注}
}

\def\equationautorefname{式}
\def\footnoteautorefname{脚注}
\def\itemautorefname{项}
\def\figureautorefname{图}
\def\tableautorefname{表}
\def\partautorefname{篇}
\def\appendixautorefname{附录}
\def\chapterautorefname{章}
\def\sectionautorefname{节}
\def\subsectionautorefname{小节}
\def\subsubsectionautorefname{小节}
\def\paragraphautorefname{段落}
\def\subparagraphautorefname{子段落}
\def\FancyVerbLineautorefname{行}
\def\theoremautorefname{定理}